\documentclass[12pt]{article}
\usepackage{geometry}
\geometry{a4paper}
\usepackage[utf8]{inputenc}
\usepackage{amsmath,amssymb}
\usepackage{hyperref} % verlinkungen
\usepackage{textcomp}
\usepackage{flafter}
\usepackage{booktabs}
\usepackage{array}
\usepackage{paralist}
\usepackage{dsfont}
\usepackage{color}
\usepackage{bbold}
\usepackage{float}
\usepackage[font=small,labelfont=bf]{caption}
\usepackage{subcaption}
\usepackage{fancyhdr}
\setlength{\headheight}{15.2pt}
\pagestyle{fancy}
\usepackage{listings} 
\usepackage{pict2e}
\usepackage{xfrac}
\usepackage[english]{babel}
\usepackage{mathtools}
\usepackage{graphicx}


%%%%%%%%         EIGENEBEFEHLE  %%%%%%%%%
\newcommand{\ddt}{\frac{\partial}{\partial{t}}}
\newcommand{\dnach}[1]{\frac{\partial}{\partial{#1}}}
\newcommand{\ddnach}[2]{\frac{\partial{#1}}{\partial{#2}}}
\newcommand{\dddnach}[3]{\frac{\partial^2{#1}}{\partial{#2} \partial{#3}}}
\newcommand{\dznach}[2]{\frac{\partial^2{#1}}{\partial{#2}^2}}
\newcommand{\vnabla}{\vec{\nabla}}
\newcommand{\evec}[1]{\mathbf{\hat{#1}}}
\newcommand{\bra}[1]{\langle{#1}|}
\newcommand{\ket}[1]{|{#1}\rangle}
\newcommand{\bracket}[2]{\langle{#1}|{#2}\rangle}
\newcommand{\up}{\uparrow}
\newcommand{\down}{\downarrow}
\newcommand{\updown}{\uparrow\downarrow}
\newcommand{\downup}{\downarrow\uparrow}
\newcommand{\upup}{\uparrow\uparrow}
\newcommand{\downdown}{\downarrow\downarrow}
\newcommand{\sandwich}[2]{\bra{#1} #2 \ket{#1}}
\newcommand{\fsqrt}[2]{\sqrt{\frac{#1}{#2}}}

\definecolor{mygreen}{rgb}{0,0.6,0}
\definecolor{mygray}{rgb}{0.5,0.5,0.5}
\definecolor{mymauve}{rgb}{0.58,0,0.82}

\lstset{%
  backgroundcolor=\color{white},   % choose the background color; you must add \usepackage{color} or \usepackage{xcolor}
  basicstyle=\footnotesize,        % the size of the fonts that are used for the code
  breakatwhitespace=false,         % sets if automatic breaks should only happen at whitespace
  breaklines=true,                 % sets automatic line breaking
  captionpos=b,                    % sets the caption-position to bottom
  commentstyle=\color{mygreen},    % comment style
  %deletekeywords={...},            % if you want to delete keywords from the given language
  %escapeinside={\%*}{*)},          % if you want to add LaTeX within your code
  extendedchars=true,              % lets you use non-ASCII characters; for 8-bits encodings only, does not work with UTF-8
  frame=single,                    % adds a frame around the code
  keepspaces=true,                 % keeps spaces in text, useful for keeping indentation of code (possibly needs columns=flexible)
  keywordstyle=\color{blue},       % keyword style
  language=C,                 % the language of the code
  %morekeywords={*,...},            % if you want to add more keywords to the set
  numbers=left,                    % where to put the line-numbers; possible values are (none, left, right)
  numbersep=5pt,                   % how far the line-numbers are from the code
  numberstyle=\tiny\color{mygray}, % the style that is used for the line-numbers
  rulecolor=\color{black},         % if not set, the frame-color may be changed on line-breaks within not-black text (e.g. comments (green here))
  showspaces=false,                % show spaces everywhere adding particular underscores; it overrides 'showstringspaces'
  showstringspaces=false,          % underline spaces within strings only
  showtabs=false,                  % show tabs within strings adding particular underscores
  stepnumber=2,                    % the step between two line-numbers. If it's 1, each line will be numbered
  stringstyle=\color{mymauve},     % string literal style
  tabsize=2,                       % sets default tabsize to 2 spaces
  %title=\lstname                   % show the filename of files included with \lstinputlisting; also try caption instead of title
}

\lhead[\ ]{\ }
\chead[\ ]{\ }
\rhead[\ ]{\ }

\lfoot[\ ]{\ }
\cfoot[\ ]{\ }
\rfoot[\ ]{\ }

%%%%%%%%%%%%%%%%%%%%%%%%%%%%%%%%%%%%%%%%%%%%

\begin{document}
\title{Temperature Measurements in Optical Tweezer Experiments}
\author{Mathias H\"old, BSc.}
\date{2016}
\maketitle
\thispagestyle{empty}
\newpage
\section{Introduction}
%something about computer science in general, optical tweezers and the need to study the
%systems on computers yada yada





\newpage
\section{Motivation}
%introduction of the experiment, the problem and the idea






\newpage
\section{Simulation}
%simulation techniques and general concept of the simulation itself, used methods and so on 
The problem at hand can be studied on an atomic level with the use of computer simulation. There is a variety of methods for computer simulations
that are widely used, one of which being Molecular Dynamics (MD) simulations. The following section will give a brief overview of the concepts of this
method, which is followed by the application to the simulation of the experiment.

\subsection{Molecular Dynamics}
Molecular Dynamics\cite{Frenkel2001} simulations is a technique for simulating, as the name suggests, the dynamics of a classical many-body system. In this case,
classical means, that the trajectories of the individual particles are calculated using classical mechanics rather then quantum mechanics. For
relatively big atoms/molecules this is a very good approximation, whereas for systems consisting of hydrogen or helium the effects of quantum
mechanics cannot be neglected and other methods (such as ab-initio simulation) has to be used.\\
The dynamics of the system are obtained by solving Newton's equations of motion for every particle. 


%\subsection{The Glass Nanoparticle}
%The glass particle from the experiment will be represented by a system of 864 particles, aligned regularly on a FCC (face centered cubic) lattice.
%This alignment is achieved by defining a minimum cell, containing points:
%\begin{eqnarray*}
    %p_1 &=& \{0,0,0\}\\
    %p_2 &=& \{0.5,0.5,0\}\\
    %p_3 &=& \{0.5,0,0.5\}\\
    %p_4 &=& \{0,0.5,0.5\}
%\end{eqnarray*}
%The whole system is then  created by copying this unit cell.\\
%The interaction between the atoms is modeled by the Lennard-Jones potential. It has the form
%\begin{equation}
    %U(r) = 4\varepsilon\left[\left(\frac\sigma r\right)^{12} - \left(\frac\sigma r\right)^6\right].
%\end{equation}
%To simplify the equation and computation of the potential and other quantities, like the forces or pressure, it is useful to introduce reduced units.
%In general, reduced units BLABLABLA THINK OF GOOD TEXT HEREJKA\\
%The basic units in a system with Lennard-Jones interaction are length ($\sigma$), energy ($\varepsilon$ or $\varepsilon / k_B$) and mass ($m$). Every 
%quantity can now be written in terms of this units, so they become reduced quantities, denotet by an asterisk. The most important ones are:
%\begin{eqnarray*}
    %r^* &=& r/\sigma\\
    %T^* &=& k_B/\varepsilon \ T\\
    %U^* &=& U/\varepsilon \\
    %P^* &=& P \sigma^3/\varepsilon 
%\end{eqnarray*}
%With the above introduced reduced units, the Lennard-Jones potential can be written as
%\begin{equation}
    %U(r^*) = 4\left[{r^*}^{-12} - {r^*}^{-6}\right].
%\end{equation}
%Since this is the most practical form, it will be used without the asterisk from here on.\\
%The Lennard-Jones potential is an additive pair-potential, so the total energy of the system can be calculated by summing over all pairs of atoms:
%\begin{equation}
    %U_{\text{tot}} = \sum_{i=1}^{N-1}\sum_{j=i+1}^N4\left[{r_{ij}}^{-12} - {r_{ij}}^{-6}\right]
%\end{equation}
%where $r_{ij}$ denotes the distance between atom $i$ and $j$. \\
%The Velocity-Verlet algorithm uses the forces to calculate the positions and velocities for the next timestep, we need to calculate the derivative of
%the potential energy:
%\begin{eqnarray}
    %F_{x} &=& -\frac{\partial}{\partial x} U(r) \nonumber\\
                %&=& -\frac{\partial}{\partial x} 4\left[{r}^{-12} - {r}^{-6}\right] \nonumber\\
                %&=& -4 \left[(-12){r}^{-13} - (-6){r}^{-7}\right] \frac{\partial r}{\partial x} \nonumber\\
                %&=& 48 \left[r^{-13} - 0.5 \ r^{-7}\right] \frac{x}{r} \nonumber\\
    %\label{eq:ljforce} &=& 48 \left[r^{-14} - 0.5 \ r^{-8}\right] x
%\end{eqnarray}
%This is the component of the force in x-direction -- the other components are calculated analogously.






\newpage
\section{Results}
% results with different parameters, graphs, etc.





\newpage
\section{Conclusion}
% what does this all tell us? do we need to study more of this? 






\newpage
\bibliography{references}
\bibliographystyle{plain}

\end{document}
