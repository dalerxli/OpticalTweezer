\documentclass[12pt]{article}
\usepackage{geometry}
\geometry{a4paper}
\usepackage[utf8]{inputenc}
\usepackage{amsmath,amssymb}
\usepackage{hyperref} % verlinkungen
\usepackage{textcomp}
\usepackage{flafter}
\usepackage{booktabs}
\usepackage{array}
\usepackage{paralist}
\usepackage{dsfont} \usepackage{color}
\usepackage{bbold}
\usepackage{float}
\usepackage[font=scriptsize,labelfont=bf]{caption}
%\usepackage[font=footnotesize,labelfont=bf]{caption}
\usepackage{subcaption}
\usepackage{fancyhdr}
\setlength{\headheight}{15.2pt}
\pagestyle{fancy}
\usepackage{listings} 
\usepackage{pict2e}
\usepackage{xfrac}
\usepackage[english]{babel}
\usepackage{mathtools}
\usepackage{graphicx}



%%%%%%%%         EIGENEBEFEHLE  %%%%%%%%%
\newcommand{\ddt}{\frac{\partial}{\partial{t}}}
\newcommand{\dnach}[1]{\frac{\partial}{\partial{#1}}}
\newcommand{\ddnach}[2]{\frac{\partial{#1}}{\partial{#2}}}
\newcommand{\dddnach}[3]{\frac{\partial^2{#1}}{\partial{#2} \partial{#3}}}
\newcommand{\dznach}[2]{\frac{\partial^2{#1}}{\partial{#2}^2}}
\newcommand{\vnabla}{\mathbf{\nabla}}
\newcommand{\emathbf}[1]{\mathbf{\hat{#1}}}
\newcommand{\bra}[1]{\langle{#1}|}
\newcommand{\ket}[1]{|{#1}\rangle}
\newcommand{\bracket}[2]{\langle{#1}|{#2}\rangle}
\newcommand{\up}{\uparrow}
\newcommand{\down}{\downarrow}
\newcommand{\updown}{\uparrow\downarrow}
\newcommand{\downup}{\downarrow\uparrow}
\newcommand{\upup}{\uparrow\uparrow}
\newcommand{\downdown}{\downarrow\downarrow}
\newcommand{\sandwich}[2]{\bra{#1} #2 \ket{#1}}
\newcommand{\fsqrt}[2]{\sqrt{\frac{#1}{#2}}}
\newcommand{\GammaK}{{\Gamma_k}}

\definecolor{mygreen}{rgb}{0,0.6,0}
\definecolor{mygray}{rgb}{0.5,0.5,0.5}
\definecolor{mymauve}{rgb}{0.58,0,0.82}

\lstset{%
  backgroundcolor=\color{white},   % choose the background color; you must add \usepackage{color} or \usepackage{xcolor}
  basicstyle=\footnotesize,        % the size of the fonts that are used for the code
  breakatwhitespace=false,         % sets if automatic breaks should only happen at whitespace
  breaklines=true,                 % sets automatic line breaking
  captionpos=b,                    % sets the caption-position to bottom
  commentstyle=\color{mygreen},    % comment style
  %deletekeywords={...},            % if you want to delete keywords from the given language
  %escapeinside={\%*}{*)},          % if you want to add LaTeX within your code
  extendedchars=true,              % lets you use non-ASCII characters; for 8-bits encodings only, does not work with UTF-8
  frame=single,                    % adds a frame around the code
  keepspaces=true,                 % keeps spaces in text, useful for keeping indentation of code (possibly needs columns=flexible)
  keywordstyle=\color{blue},       % keyword style
  language=C,                 % the language of the code
  %morekeywords={*,...},            % if you want to add more keywords to the set
  numbers=left,                    % where to put the line-numbers; possible values are (none, left, right)
  numbersep=5pt,                   % how far the line-numbers are from the code
  numberstyle=\tiny\color{mygray}, % the style that is used for the line-numbers
  rulecolor=\color{black},         % if not set, the frame-color may be changed on line-breaks within not-black text (e.g. comments (green here))
  showspaces=false,                % show spaces everywhere adding particular underscores; it overrides 'showstringspaces'
  showstringspaces=false,          % underline spaces within strings only
  showtabs=false,                  % show tabs within strings adding particular underscores
  stepnumber=2,                    % the step between two line-numbers. If it's 1, each line will be numbered
  stringstyle=\color{mymauve},     % string literal style
  tabsize=2,                       % sets default tabsize to 2 spaces
  %title=\lstname                   % show the filename of files included with \lstinputlisting; also try caption instead of title
}

\lhead[\ ]{\ }
\chead[\ ]{\ }
\rhead[\ ]{\ }

\lfoot[\ ]{\ }
\cfoot[\ ]{\ }
\rfoot[\ ]{\ }

%%%%%%%%%%%%%%%%%%%%%%%%%%%%%%%%%%%%%%%%%%%%

\begin{document}
\title{Expos\'e - Simulation of Nano Particles in a Laser Trap}
\author{Mathias H\"old, BSc.}
\maketitle
\thispagestyle{empty}
\section{Konzept}
In dieser Masterarbeit wird mit Hilfe von Methoden der Computational Physics ein Experiment simuliert, in dem Nanopartikel in einer Laserfalle
gefangen werden. Das Hauptaugenmerk dabei liegt auf der Interaktion des Nanopartikels mit dem umgebenden Gas und dem Einfluss des Lasers auf die
Bewegung des Nanopartikels in der Falle. Die benutzte Technik in dieser Simulation nennt sich ,,molecular dynamics'' 
und wird dazu verwendet, um die zeitliche Entwicklung des Zustandes des Nanopartikels und des Gases zu berechnen.\\
Das Experiment besteht aus einem Nanopartikel, in diesem Fall eine Siliziumdioxid-Kugel mit einem Durchmesser von $r \approx 75 nm$, das in von einem
Laser durch Gradientenkraft lokalisiert wird. Das Nanopartikel befindet sich in einer Vakuumkammer mit variablem Druck.\\
Die Gasteilchen, die das Nanopartikel umgeben, interagieren nicht miteinander, wenn die mittlere freie Wegl\"ange der Gasteilchen viel gr\"o\ss er ist
als der Radius des Nanopartikels. Das bedeutet, dass die Interaktion nur zwischen den Gas Teilchen und dem Nanopartikel in der Falle stattfindet. 
Dadurch entstehen zwei W\"armereservoirs des umgebenden -- ein ,,kaltes'', bestehend aus Teilchen vor der Interaktion mit dem Nanopartikel und ein
,,hei\ss es'', bestehend aus Teilchen nach der Interaktion mit dem Nanopartikel.\\
Dem Nanopartikel selbst sind auch zwei Temperaturen zuzuweisen. Da es sich in einer Laserfalle befindet, absorbiert das Nanopartikel die Energie des
Lasers in Form von W\"arme, die an die innere Energie der Atome, aus dem das Nanopartikel besteht, abgegeben wird. Die Temperatur, die aus dieser
inneren Energie abzuleiten ist, wird ,,Oberfl\"achentemperatur'' genannt. Die zweite Temperatur, die dem Nanopartikel zuzuordnen ist, ist seine
Schwerpunktstemperatur -- jene Temperatur, die aus der Geschwindigkeit der Schwerpunktsbewegung berechnet wird. Durch die Lokalisierung in der
Laserfalle ist wird diese Bewegung minimiert. \\
Die Simulation dieses Experiments wird durch Modellierung der Einzelnen Bestandteile durchgef\"uhrt. Das Nanopartikel wird durch einen FCC Kristall
approximiert, in dem die Teilchen miteinander \"uber ein Lennard-Jones Potential miteinander wechselwirken. Der Laser wird durch zwei unterschiedliche
Teile modelliert. Zum einen wird die durch den Laser abgegebene und das Nanopartikel aufgenommene Energie durch den so genannten ,,eHEX'' -- enhanced
heat exchange algorithm -- modelliert. Die Lokalisierung des Nanopartikels durch den Laser wird durch ein harmonisches Potential modelliert, das auf
den Schwerpunkt wirkt. Das umgebende Gas wird durch einen Barostat modelliert, der in diesem Fall eher eine Funktion als Thermostat einnimmt. Die
Bewegung der einzelnen Atome wird durch den velocity Verlet Algorithmus berechnet.\\
Die Fragestellung, der in dieser Arbeit nachgegangen wird ist: Hat die erh\"ohte Temperatur der inneren Freiheitsgrade des Nanopartikels einen
Einfluss auf dessen Schwerpunktsbewegung?

\section{Vorl\"aufiges Inhaltsverzeichnis}
Die Arbeit wird in folgende Kapitel unterteilt sein:
\begin{itemize}
    \item{Introduction}
    \item{Motivation}
    \item{Simulation}
    \item{Results}
    \item{Conclusion}
\end{itemize}

\section{Auswahlbibliographie}
\nocite{*} 
\bibliography{references}
\bibliographystyle{unsrt}

\end{document}
