\documentclass[12pt]{article}
\usepackage{geometry}
\geometry{a4paper}
\usepackage[utf8]{inputenc}
\usepackage{amsmath,amssymb}
\usepackage{hyperref} % verlinkungen
\usepackage{textcomp}
\usepackage{flafter}
\usepackage{booktabs}
\usepackage{array}
\usepackage{paralist}
\usepackage{dsfont} \usepackage{color}
\usepackage{bbold}
\usepackage{float}
\usepackage[font=scriptsize,labelfont=bf]{caption}
%\usepackage[font=footnotesize,labelfont=bf]{caption}
\usepackage[font=scriptsize]{subcaption}
\usepackage{fancyhdr}
\setlength{\headheight}{15.2pt}
\pagestyle{fancy}
\usepackage{listings} 
\usepackage{pict2e}
\usepackage{xfrac}
\usepackage[english]{babel}
\usepackage{mathtools}
\usepackage{graphicx}
\usepackage{pdfpages}



%%%%%%%%         EIGENEBEFEHLE  %%%%%%%%%
\newcommand{\ddt}{\frac{\partial}{\partial{t}}}
\newcommand{\dnach}[1]{\frac{\partial}{\partial{#1}}}
\newcommand{\ddnach}[2]{\frac{\partial{#1}}{\partial{#2}}}
\newcommand{\dddnach}[3]{\frac{\partial^2{#1}}{\partial{#2} \partial{#3}}}
\newcommand{\dznach}[2]{\frac{\partial^2{#1}}{\partial{#2}^2}}
\newcommand{\vnabla}{\mathbf{\nabla}}
\newcommand{\emathbf}[1]{\mathbf{\hat{#1}}}
\newcommand{\bra}[1]{\langle{#1}|}
\newcommand{\ket}[1]{|{#1}\rangle}
\newcommand{\bracket}[2]{\langle{#1}|{#2}\rangle}
\newcommand{\up}{\uparrow}
\newcommand{\down}{\downarrow}
\newcommand{\updown}{\uparrow\downarrow}
\newcommand{\downup}{\downarrow\uparrow}
\newcommand{\upup}{\uparrow\uparrow}
\newcommand{\downdown}{\downarrow\downarrow}
\newcommand{\sandwich}[2]{\bra{#1} #2 \ket{#1}}
\newcommand{\fsqrt}[2]{\sqrt{\frac{#1}{#2}}}
\newcommand{\GammaK}{{\Gamma_k}}

\definecolor{mygreen}{rgb}{0,0.6,0}
\definecolor{mygray}{rgb}{0.5,0.5,0.5}
\definecolor{mymauve}{rgb}{0.58,0,0.82}

\lstset{%
  backgroundcolor=\color{white},   % choose the background color; you must add \usepackage{color} or \usepackage{xcolor}
  basicstyle=\footnotesize,        % the size of the fonts that are used for the code
  breakatwhitespace=false,         % sets if automatic breaks should only happen at whitespace
  breaklines=true,                 % sets automatic line breaking
  captionpos=b,                    % sets the caption-position to bottom
  commentstyle=\color{mygreen},    % comment style
  %deletekeywords={...},            % if you want to delete keywords from the given language
  %escapeinside={\%*}{*)},          % if you want to add LaTeX within your code
  extendedchars=true,              % lets you use non-ASCII characters; for 8-bits encodings only, does not work with UTF-8
  frame=single,                    % adds a frame around the code
  keepspaces=true,                 % keeps spaces in text, useful for keeping indentation of code (possibly needs columns=flexible)
  keywordstyle=\color{blue},       % keyword style
  language=C,                 % the language of the code
  %morekeywords={*,...},            % if you want to add more keywords to the set
  numbers=left,                    % where to put the line-numbers; possible values are (none, left, right)
  numbersep=5pt,                   % how far the line-numbers are from the code
  numberstyle=\tiny\color{mygray}, % the style that is used for the line-numbers
  rulecolor=\color{black},         % if not set, the frame-color may be changed on line-breaks within not-black text (e.g. comments (green here))
  showspaces=false,                % show spaces everywhere adding particular underscores; it overrides 'showstringspaces'
  showstringspaces=false,          % underline spaces within strings only
  showtabs=false,                  % show tabs within strings adding particular underscores
  stepnumber=2,                    % the step between two line-numbers. If it's 1, each line will be numbered
  stringstyle=\color{mymauve},     % string literal style
  tabsize=2,                       % sets default tabsize to 2 spaces
  %title=\lstname                   % show the filename of files included with \lstinputlisting; also try caption instead of title
}

\lhead[\ ]{\leftmark }
\chead[\ ]{\ }
\rhead[\ ]{\rightmark }

\lfoot[\ ]{\ }
\cfoot[\ ]{\thepage }
\rfoot[\ ]{\ }

%%%%%%%%%%%%%%%%%%%%%%%%%%%%%%%%%%%%%%%%%%%%

\begin{document}

\section{Einleitung}
In meiner Masterarbeit beschäftigte ich mich mit der Simulation eines Nano Teilchens in einer Laserfalle. Diese Arbeit verbindet damit 
zwei Teilbereiche der Physik miteinander: Computational Physics und Optische Fallen, die ich beide kurz vorstellen möchte.\\
Die Computational Physics (oder auch Computergestützte Physik) verwendet numerische Methoden, um physikalische Prozesse auf dem Computer
zu simulieren. Ein Meilenstein, der genannt werden sollte, ist die Arbeit von Metropolis et al., die im Jahre 1953 mit Hilfe des Computers
MANIAC (Mathematical Analyzer Numerical Integrator And Computer) in Los Alamos die Zustandsgleichung eines Fluids berechneten. Die dort 
benutzte Methode, der Metropolis Monte Carlo Algorithmus, ist bis heute ein sehr wichtiger Bestandteil der Compuational Physics. Seit
1953 hat sich nicht nur die Rechenleistung von Computern enorm verbessert, sondern es wurden auch Zahlreiche neue Methoden zur Simulation 
von physikalischen Prozessen entwickelt, wie zum Beispiel Transition Path Sampling, Finite-Elemente-Methode und Molekulardynamik-Simulation. 
Letztere wurde in dieser Arbeit verwendet und wird später etwas detaillierter beschrieben.\\
Laserfallen, auch Optische Pinzette (englisch \textit{optical tweezers} genannt) ist eine experimentelle Methode, bei der Objekte durch die 
Gradientenkraft eines fokussierten Laserstrahls lokalisiert werden. Die Größe der Objekte reicht von der subatomaren (Kühlung einzelner Atome) bis zur
Mikrometer-Skala (Zellen). Die Idee für diese Methode entstand schon zu Beginn des 20. Jahrhunderts, als Lebedev und Nichols und Hull die 
Existenz von Strahlungsdruck experimentell nachweisen konnten. Da die Technik zu der Zeit noch nicht so ausgereift war, wurde die Idee erst 
1970 wieder durch Ashkin aufgegriffen. Dieser verwendete Laser, um die Bewegung von Atomen und Mikrometergroßen Objekten zu beeinflussen.\\

\section{Motivation}
Die Motivation für diese Arbeit ist ein Optical Tweezer Experiment, das von Jan Gieseler, Romain Quidant, Christoph Dellago und Lukas Novotny
durchgeführt wurde. Ziel des Experiments war der Nachweis des Fluktuationstheorems.\\
Im Experiment befindet sich eine Nanokugel mit einem Durchmesser von etwa 75 Nanometer und einer Masse von $3 \times 10^{-18}$ kg in einer
Laserfalle in einer Vakuumkammer. Die Kugel wird durch die Gradientenkraft des Lasers lokalisiert. In der Laserfalle fluktuiert die Position 
der Kugel in alle 3 Raumrichtungen. Die Fluktuation in jede Richtung ist von den anderen Richtungen entkoppelt und kann daher in einer 
eindimensionalen Bewegungsgleichung, der Langevin-Gleichung, beschrieben werden:
\begin{equation}
    \label{eq:langevin}
    \ddot{x} + \Gamma_0 \dot{x} + \Omega^2_0x = \frac 1 m \left(F_\text{fluct} + F_\text{ext}\right)
\end{equation}
Auf der linken Seite: 
$x$ bezeichnet den Ort des Teilchens, $\dot{x}$ und $\ddot{x}$ die Geschwindigkeit und Beschleunigung, $\Gamma_0$ ist der Reibungskoeffizient
und $\Omega_0$ ist die Winkelfrequenz, die die Fluktuation entlang der gewählten Raumrichtung beschreibt. Auf der rechten Seite befinden sich 
zwei Kräfte. Die erste, $F_\text{fluct}$ beschreibt die stochastische Kraft, die durch die Stöße mit dem umgebenden Gas verursacht wird. 
Diese Kraft ist gegeben durch:
\begin{equation}
    \label{eq:ffluct}
    F_\text{fluct} = \sqrt{2m\Gamma_0k_BT_0} \ \xi\left(t\right)
\end{equation}
Hier ist $T_0$ die Temperatur des Wärmereservoirs, also des umgebenden gases, $k_B$ ist die Boltzmann-Konstante und $\xi(t)$ ist weißes 
Rauschen für das gilt $\left\langle\xi(t)\right\rangle= 0$ und $\left\langle\xi(t)\xi(t')\right\rangle= \delta(t-t')$, d.h. es handelt sich um 
eine randomisierte Kraft. Der Term $\Gamma_0$ taucht in der Formel wegen des Fluktuations-Dissipations-Theorems auf, das die Dämpfungsrate
und die stochastische Kraft verbindet. \\
Die andere Kraft, $F_\text{ext}$ ist eine externe, zeitabhängige Kraft, die das Teilchen in einen stationären Nichtgleichgewichtszustand zu bringen. 
Die Kraft wird zum Zeitpunkt $t = t_\text{off}$ deaktiviert und die Relaxation des Teilchens wird gemessen. Diese Messung wird 
durchgeführt, um das Fluktuationstheorem zu untersuchen, das geschrieben werden kann als:
\begin{equation}
    \label{eq:fluctuationtheorem}
    \frac{p(-\Delta S)}{\Delta S} = e^{-\Delta S}
\end{equation}
wobei $\Delta S$ die relative Entropieänderung ist, gegeben durch:
\begin{equation}
    \Delta S = \beta_0 Q + \Delta \phi
\end{equation}
$Q$ bezeichnet die vom Wärmereservoir absorbierte Wärme, $\beta$ ist die reziproke Temperatur bei der die Wärme absorbiert wird
und $\Delta \phi$ ist die Differenz der trajektoriebasierten Entropie. Das Fluktuationstheorem beschreibt die 
Wahrscheinlichkeit für Verletzung des 2. Hauptsatzes für finite (kleine) Zeitintervalle. Das ist ein statistischer Effekt und steht nicht
im Widerspruch zum 2. Hauptsatz.\\
Die Untersuchung des Fluktuationstheorems wird bei Relaxationsprozessen von zwei unterschiedlichen stationären Nichtgleichgewichtszuständen
durchgeführt. Der erste wird durch einen Feedback-Mechanismus erzeugt und der zweite durch eine Kombination des Feedback-Mechanismus und einem
zusätzlichen Feedback-Drives. Die Ergebnisse zeigen, dass beide Zustände beim Relaxationsprozess dem Fluktuationstheorem genügen, wodurch 
das Theorem für diesen Fall experimentell bestätigt wird.\\
J. Millen, T. Deesuwan, P. Barker und J. Anders führten ein ähnliches Experiment durch (ohne den Feedback Mechanismus), wo sie die Temperatur
des umgebenden Gases untersuchten.\\
Während in dem Modell von Gieseler et al. nur eine Temperatur $T_0$ verwendet wird, stellen Millen et al. ein Modell mit vier unterschiedlichen
Temperaturen auf: die Temperatur des einströmenden und ausströmenden Gases, der Oberfläche und des Schwerpunkts des Teilchens.\\
Das Nanoteilchen in der Falle absorbiert Wärme vom Laser, was zu einer erhöhten Oberflächentemperatur führt, die im Allgmeinen höher ist 
als die Temperatur des einströmenden Gases. Die eiströmenden Gasteilchen wechselwirken mit dem Teilchen in der Falle, nehmen dabei Energie 
auf und verlassen das Teilchen mit einer höheren Temperatur. Im Modell wird angenommen, dass die Gasteilchen nur mit dem Nanopartikel in der
Falle und nicht untereinander interagieren, was zu der Entstehung von zwei Wärmereservoirs mit unterschiedlicher Temperatur führt.\\
Diese Unterschiedlichen Temperaturen führen zu einer Modifikation der eindimensionalen Langevin-Gleichung:
\begin{equation}
    M\ddot{x}(t) + M\left(\Gamma^\text{imp}+\Gamma^\text{em}\right)\dot{x}(t) + M\omega^2x(t) =F^\text{imp}+F^\text{em}
\end{equation}
x, $\dot{x}$ und $\ddot{x}$ sind Position, Geschwindigkeit und Beschleunigung des Nanoteilchens, M ist die Masse des Nanopartikels, 
$\omega$ ist die Frequenz der Falle, $\Gamma^\text{Imp}$ und $\Gamma^\text{Em}$ sind die Dämpfungsterme für das ein- und ausgehende Gas
und $F^\text{Imp}$ und $F^\text{Em}$ sind die zugehörigen Noise-Terme.\\
Im Experiment zeigen Millen et al., dass die Laserintensität und Teilchengröße einen Einfluss auf die Temperatur der ausgehenden Gastteilchen hat.\\
Man sieht also, dass im Modell für die Bewegung des Teilchens in der Falle mehr Temperaturen als nur die Temperatur des umgebenden Gases
berücksichtigt werden sollen. In dieser Masterarbeit beschäftigte ich mich daher mit der Frage: Welchen Einfluss haben Laserintensität
und die Temperatur des eingehenden Gases auf die Bewegung des Nanoteilchens in der Falle?



\section{Simulation}
Um das Experiment zu simulieren, müssen die einzelnen Bestandteile des Experiments modelliert werden. Das Nanoteilchen in der Falle
wird durch ein System aus Atomen in einem FCC Gitter approximiert, die Gradientenkraft des Lasers durch ein harmonisches Potential, 
die Absorption der Laserenergie im Teilchen durch den eHEX Algorithmus und das umgebende Gas durch einen Thermostaten mit einem idealen Gas
als Druckmedium. Die Dynamik des Systems wird mit Hilfe der Methode der Molekulardynamik-Simulation berechnet.\\
\subsection{Molekulardynamik - Die Idee}
Molekulardynamik Simulationen werden verwendet um die Dynamik klassischer atomarer oder molekularer Vielteilchensysteme zu berechnen. 
Klassisch
in diesem Sinne bedeutet, dass die Trajektorien mit Hilfe von Modellen der klassischen Physik und nicht der Quantenmechanik 
berechnet werden. Für große Systeme liefert das sehr gute Ergebnisse. Für kleine Systeme wie Wasserstoff oder Helium sind quantenmechanische
Effekte nicht mehr vernachlässigbar und daher sollten andere Methoden dafür verwendet werden.

\subsection{Velocity-Verlet Algorithmus}
Mikroskopisch gesehen beschreitet das System eine Trajektorie im Phasenraum, 
wo jeder Punkt einem Set von Positionen und Impulsen entspricht. Die Verbindung zwischen
zwei Punkten entspricht der Entwicklung von einem Zustand zum anderen. Diese Entwicklung gilt es zu simulieren. Dazu wird 
die Methode der finiten Differenzen gewählt. Die Trajektorie im Phasenraum wird in Abschnitte mit Länge $\Delta t$ geteilt und die 
Bewegungsgleichungen für jedes segment werden berechnet.\\
Der Algorithmus, der hier verwendet wird und eine sehr zuverlässige Wahl ist, ist der Velocity Verlet Algorithmus. Er basiert auf der 
Taylorreihenentwicklung der Koordinaten und Geschwindigkeiten. Der Algorithmus kann durch 2 simple Formeln dargestellt werden:
\begin{equation}
    \label{eq:velocityverlet}
    \begin{aligned}
        \mathbf{r}_i(t+\Delta t) = \mathbf{r}_i(t) + {\mathbf{v}}_i(t) \Delta t + \frac1{2m} {\mathbf{F}}_i(t) \Delta t^2\\
        \mathbf{v}_i(t+\Delta t) = \mathbf{v}_i(t) + \frac1{2m} \Big[\mathbf{F}_i(t) + \mathbf{F}_i(t+\Delta t)\Big] \Delta t
    \end{aligned}
\end{equation}
Wo $r_i$ die Position des Teilchens $i$ ist, $v_i$ ist seine Geschwindigkeit und $F_i$ ist die auf das Teilchen wirkende Kraft, 
$\Delta t$ ist der benützte Zeitschritt. 
Umgesetzt wird der Algorithmus durch folgende Schritte
\begin{enumerate}
    \item Berechne die Kräfte zwischen den Teilchen (nur erster Schritt)
    \item Berechne neue Positionen
    \item Berechne neue Geschwindigkeiten 
    \item Berechne Kräfte mit neuen Positionen 
    \item Berechne neue Geschwindigkeit mit neuen Kräften -> benütze diese für Schritt 2 und setze dort fort
\end{enumerate}

\subsection{Das Nanoteilchen}
Um das Nanoteilchen in der Falle zu modellieren wird ein System aus N Teilchen verwendet, die durch das Lennard-Jones Potential 
wechselwirken:
\begin{equation}
    \label{eq:lj}
    U(r) = 4\varepsilon\left[\left(\frac\sigma r\right)^{12} - \left(\frac\sigma r\right)^6\right]
\end{equation}
hier ist $\varepsilon$ die Potentialtiefe, $\sigma$ die Distanz bei der das Potential gleich Null ist und r ist der intermolekulare Abstand.\\
Um Rechnungen zu vereinfachen werden so genannte reduzierte Einheiten verwendet: $\sigma$ als Einheit der Länge, $\varepsilon$ als Einheit 
der Energie und die molekulare Masse $m$ als Masse. Üblicherweise werden reduzierte Variablen mit einem Stern versehen. Da aber hier reduzierte 
Einheiten ständig verwendet werden, wird der Stern weggelassen.
%Daraus können andere Einheiten abgeleitet werden. Zum Beispiel
%\begin{itemize}
    %\item {distance:} $r^* = r/\sigma$
    %\item {potential energy:} $U^* = U/\varepsilon$
    %\item {temperature:} $T^* = k_B T/\varepsilon$
    %\item {time:} $t^* = t\sqrt{\varepsilon/(m\sigma^2)}$
    %\item {pressure:} $P^* = P\sigma^3/\varepsilon$
    %\item {density:} $\rho^* = \rho \sigma^3$
%\end{itemize}
Das Lennard Jones Potential kann damit geschrieben werden als:
\begin{equation}
    U(r^*) = 4\left[{r}^{-12} - {r}^{-6}\right].
\end{equation}
Die zugehörige Kraft, die für den Velocity Verlet Algorithmus gebraucht wird, wird durch den negativen Gradienten des
Potentials berechnet. Als Beispiel die Kraft in x-Richtung:
\begin{eqnarray}
    F_{x} &=& -\frac{\partial}{\partial x} U(r) \nonumber\\
                &=& -\frac{\partial}{\partial x} 4\left[{r}^{-12} - {r}^{-6}\right] \nonumber\\
    \label{eq:ljforce} &=& 48 \left[r^{-14} - 0.5 \ r^{-8}\right] x
\end{eqnarray}
(y,z analog).\\
Ein wichtiger Schritt in der Simulation ist die Initialisierung der Teilchen. Für diese Arbeit wurde als 
Startkonfiguration ein kubisch flächenzentriertes Gitter (FCC, face centered cubic) gewählt. 
Die Anzahl der Teilchen in einer 
Elementarzelle eines FCC Gitters ist 4 (1/8 an jeder Ecke und 1/2 auf jeder der 6 Seitenflächen). Um einen Würfel aus solchen
Einheitszelle zu erzeugen, werden $M^3$ solcher Zellen benötigt und damit $4M^3$ Teilchen. Die Bedingung 
$N = 4M^4 = 4,32,108,256,\ldots$ führt zu den so genannten magic numbers. Um Zeit und Rechenaufwand zu sparen wurden in dieser
Arbeit $N=32$ Teilchen verwendet.\\
Die Erzeugung des Gitters wird mit Hilfe von 4 eindeutigen Punkten durchgeführt:
\begin{eqnarray*}
    p_1 &=& \{0,0,0\}\\
    p_2 &=& \{0.5 \ a,0.5 \ a,0\}\\
    p_3 &=& \{0.5 \ a,0,0.5 \ a\}\\
    p_4 &=& \{0,0.5 \ a,0.5 \ a\}
\end{eqnarray*}
wobei $a$ die Gitterkonstante des FCC Gitters ist, gegeben durch 
From the particle number $N$ and the number of FCC unit cells per edge $M$ the \textit{lattice constant} $a$ can be calculated
\begin{equation}
    a = \frac{L}{M}
\end{equation}
wobei $M$ die Anzahl der FCC Einheitszellen entlang einer Kante ist und $L$ die Seitenlänge des gesamten Systems, gegeben durch 
\begin{equation}
    L = \sqrt[3]{\frac{N}{\rho}}
\end{equation}
$\rho$ ist die Dichte des Systems. 


\subsection{Laserstrahl - Wärmequelle}
Das Teilchen in der Falle wird durch den Laserstrahl gefangen. Während die Schwerpunktsbewegung dabei lokalisert ist, absorbieren
die einzelnen Atome des Nanoteilchens Wärme vom Laser und bekommen damit eine hohe interne Temperatur.\\
Um dieses Verhalten zu simulieren wird anstatt eines Thermostaten ein Algorithmus verwendet, der Wärmeaustausch als Grundlage hat.
Dieser Algorithmus heißt eHEX.\\
Der grundsätzliche Aufbau sieht folgendermaßen aus: Im Gesamtsystem werden Regionen festgelegt, die entweder als Wärmequelle oder 
Wärmesenken dienen. Diese Regionen werden mit $\Gamma_k$ bezeichnet und haben korrespondierende Mengen von ausgetauschter Wärme $\Delta Q_k$. 
Negatives $\Delta Q_k$ bedeutet, dass die Region Wärme von der Umgebung aufnimmt und positives $\Delta Q_k$ bedeutet, dass die Region 
Wärme an die Umgebung abgibt. Die Teilchen in der Simulationsbox $\Gamma$ und in den ,,Wärme''-Regionen besitzen 
Schwerpunktsgeschwindigkeiten $v_\Omega$ und $v_{\Gamma_k}$. Die Änderung der Energie in den Regionen wird vorgenommen durch das Reskalieren 
der Geschwindigkeiten mit einem Faktor $\xi_k$ und einem Shift:
\begin{equation}
    \mathbf{v}_i \rightarrow \mathbf{\bar{v}}_i = \xi_k \mathbf{v}_i + (1-\xi_k)\mathbf{v}_{\Gamma_k}
\end{equation}
\begin{equation}
    \xi_k = \sqrt{1+\frac{\Delta Q_{\Gamma_k}}{\mathcal{K}_{\Gamma_k}}}
\end{equation}
wobei $\Delta Q_{\Gamma_k}$ die ausgetauschte Wärme in der Region ist und $\mathcal{K}_{\Gamma_k}$ die kinetische (Ruhe)-Energie der Region:
\begin{equation}
    \label{eq:kineticK}
    \mathcal{K}_{\Gamma_k} = \sum_{i \in \gamma_k} \frac{m_i v_i^2}{2} - \frac{m_{\Gamma_k}v_{\Gamma_k}^2}{2}
\end{equation}
Um den Algorithmus durchzuführen werden noch einige Variablen benötigt, auf die ich hier nicht im Detail eingehen möchte.\\
Die Situation für die Masterarbeit ist etwas einfacher: es gibt nur eine Region, die als Wärmequelle agiert und die
Schwerpunktsgeschwindigkeit gleich Null.\\
Der Algorithmus basiert auf einem Velocity-Verlet Algorithmus mit einem Halbschritt in der Geschwindigkeit, ist jedoch etwas komplizierter. 
Deswegen werde
ich nicht alle Formeln aufschreiben, sondern die Einzelnen Schritte beschreiben:
\begin{enumerate}
\item Berechnung aller intermolekularen Kräfte (Initialisierung)
\item Reskalierung der Geschwindigkeiten -> $\bar{v}^n$
\item Halbschritt der skalierten Geschwindigkeit durch addieren der Kraft -> $\bar{v}^{n+1/2}$
\item Berechnung der skalierten Positionen mit dem Halbschritt der Geschwindigkeiten -> $\bar{r}^{n+1}$
\item Berechnung der Kräfte basierend auf den neuen Positionen
\item Halbschritt der skalierten Geschwindigkeit mit vorigem Halbschritt und neuen Kräften -> $\bar{v}^{n+1}$
\item Berechnung der Geschwindigkeit durch erneute skalierung mit aktualisiertem $\xi$
\item Berechnung der neuen Positionen und Korrektur
\end{enumerate}
Der Korrekturterm im letzten Schritt ist das Herzstück des eHEX Algorithmus. Er sorgt dafür, dass der Algorithmus keinen 
Drift in der Energie im Langzeitverhalten aufweist. 


\subsection{Laserstrahl - Lokalisierung}
Die Lokalisierung des Teilchens wird durch ein harmonisches Potential simuliert. Kraft und Potential können geschrieben werden als
\begin{eqnarray}
    \mathbf{F} &=& -k \Big[\mathbf{r}_\text{COM}-\mathbf{x}_0\Big]\\
    U &=& \frac12 k \Big[\mathbf{r}_\text{COM}-\mathbf{x}_0\Big]^2
\end{eqnarray}
wobei $r_\text{COM}$ die Position des Schwerpunkts, $x_0$ die Position des Minimums des Potentials und k die Federkonstante ist. Die 
Schwerpunktsposition wird berechnet durch  
\begin{equation}
    \mathbf{r}_\text{COM} = \frac1N\sum_{i=1}^N \mathbf{r}_i
\end{equation}
Um das Potential in die Bewegungsgleichung einfließen zu lassen wird die Kraft auf den Schwerpunkt berechnet und zu den intermolekularen 
Kräften hinzugefügt. 



\subsection{Umgebendes Gas - Thermostat}
Im Experiment ist Teilchen von einem Gas in einer Vakuumkammer umgeben. Dieses Gas wird durch einen Thermostaten/Barostaten 
simuliert, der ein ideales Gas als Druckmedium verwendet.\\
Die Gasteilchen strömen dabei in die Simulationsbox, interagieren mit dem Teilchen in der Falle und verlassen das System wieder (mit einer höheren
Temperatur). Die Gasteilchen interagieren also nur mit dem Teilchen in der Falle, nicht untereinander. Dadurch ist dieses Modell der 
Situation im Experiment sehr ähnlich.\\
Dazu wird das simulierte Teilchen mit einem Volumen umgeben, das auf der Geometrie des Systems basiert. Der Algorithmus ist so 
gestaltet, dass sich dieses Volumen im Laufe der Simulation der Geometrie des Systems anpasst. Ich habe mich entschlossen, eine simple, sich nicht
ändernde Geometrie zu wählen: einen Würfel mit Seitenlänge L, der eine Seitenlänge des LJ-Systems vom Teilchen in der Falle entfernt ist. Damit gilt
$L = 3s$. Das verwendete Potential für die Interaktion zwischen Gas und Nanoteilchen ist ein Soft-Sphere Potential
\begin{equation}
    \label{eq:softsphere}
    U(r) = \varepsilon \left(\frac{\sigma}{r}\right)^{12}
\end{equation}
wo $\varepsilon$ die Interaktionsstärke ist, $\sigma$ die Interaktionslänge und $r$ die Distanz zwischen Gasteilchen und Atom im Nanoteilchen.\\
Der Algorithmus kann für die vorliegende Situation folgendermaßen zusammengefasst werden:
\begin{enumerate}
\item Erzeuge Zufallszahl an Teilchen, die an einer Seite des umgebenden Volumens einströmen soll aus einer Poisson-Verteilung mit 
Erwartungswert 
        \begin{equation}
            %\langle N_\text{fac}\rangle = \Delta t L^2 P \left(\frac{1}{2\pi m k_B T}\right)^\frac12
            \label{eq:numberofparticles}
            \langle N_\text{fac}\rangle = \Delta t L^2 P \sqrt{\frac{1}{2\pi m k_B T}}
        \end{equation}
wobei $\Delta t$ der Zeitschritt, $L$ Seitenlänge des Umgebenden Volumens, $P$ eingestellter Druck, $m$ Masse, $k_B$ Boltzmann Konstante und 
$T$ die Temperatur.
Die Positionen der Teilchen werden über entsprechende Fläche gleichmäßig veerteilt. Die Geschwindigkeiten werden aus zwei unterschiedlichen 
Verteilungen gewürfelt. Die Komponente, die normal auf die gewählte Fläche des Volumens ist, wird aus einer Rayleigh Verteilung gewählt
        \begin{equation}
            p(v_i) = \frac{m}{k_B T}v_i \ e^{-\frac{mv_i^2}{2k_BT}}
        \end{equation}
Die restlichen Komponenten werden aus einer Maxwell-Boltzmann Verteilung gezogen.
\item Führe den ersten Velocity Verlet Schritt aus, um neue Positionen zu berechnen
\item Wenn Gasteilchen das umgebende Volumen verlassen habe, entferne sie
\item Berechne die Kräfte
\item Führe den zweiten Schritt des Velocity Verlet Algorithmus aus
\end{enumerate}














\section{Ergebnisse}
Da ich alle Modelle selbst umgesetzt habe, sollten alle Bestandteile auf Korrektheit untersucht werden.
\subsection{Nanoteilchen}
Um zu verfizieren, dass es sich um einen FCC Kristall handelt wird die Darstellung so gewählt, dass man das für den FCC Kristall typische
ABC-Muster der close packed structure sieht. 

\subsection{Velocity Verlet}
Simulationen mit der Molekulardynamik Methode finden im mikrokanonischen Ensemble statt, d.h. Volumen, Teilchenzahl und Gesamtenergie sind konstant.
Die Geschwindigkeiten der Teilchen werden bei der Initialisierung aus einer Maxwell-Boltzmann Verteilung gewürfelt. Danach müssen zwei Anpassungen 
vorgenommen werden.\\
Um zu verhindern, dass das System sich bewegt, wird die Schwerpunktsgeschwindigkeit berechnet und von allen Teilchen subtrahiert.
\begin{eqnarray}
    \mathbf{v}_\text{COM} &=& \frac1N\sum_{i=1}^N \mathbf{v}_i\\
    %\mathbf{v}_\text{cut} &=& \frac{\mathbf{v}_\text{tot}}N\\
    \mathbf{v}_i^\text{new} &=& \mathbf{v}_i - \mathbf{v}_\text{COM}
\end{eqnarray}
Da es für die Teilchen kein sehr natürlicher Zustand ist sich genau auf den Punkten in einem FCC Gitter zu befinden benötigt das System 
einige Zeitschritte zur Equilibrierung, was dazu führen kann, dass sich eine andere Temperatur einstellt, als die, die gewünscht ist. Um zu 
versichern, dass die Temperatur der gewünschten entspricht, müssen die Geschwindigkeiten der Teilchen reskaliert werden:
\begin{eqnarray}
\label{eq:rescale1}   E &=& \sum_{i=1}^N \frac{\mathbf{v}_i^2}{2} \\
 \label{eq:rescale2}   \lambda &=& \sqrt{\frac{3(N-1)T}{2E}} \\
 \label{eq:rescale3}   \mathbf{v}_i^\text{new} &=& \lambda \mathbf{v}_i  
\end{eqnarray}
Diese Reskalierung wird einige Male vorgenommen, bis das System die gewünschte Temperatur erreicht hat. Da diese Reskalierung keinem 
natürlichen internen Prozess entspricht, kann man einen Sprung in der Kurve der Gesamtenergie sehen.

\subsection{eHEX}
Der eHEX Algorithmus besteht aus vielen Schritten und vielen Variablen, was ihn sehr fehleranfällig macht. Weiters sollte der Effekt von 
verschiedenen Werten von $\Delta Q$ untersucht werden. Was wir erwarten ist eine kontinuierliche Erhöhung der
internen Temperatur des Systems. Die Rate der Erhöhung sollte mit steigendem $\Delta Q$ steigen und die Temperatur sollte sich 
für $\Delta Q = 0$ nicht verändern. Weiters sollte mit steigender interner Temperatur auch die Gesamtenergie des Systems
steigen. Diese zwei Erwartungen werden erfüllt, wie die beiden Bilder zeigen.

\subsection{Thermostat}
Der Thermostat hält das System davon ab sich zu überhitzen. Durch Interaktion von Gasteilchen mit dem Nanoteilchen in der Falle
wird Energie auf die Gasteilchen übertragen und somit stellt sich eine neue, höhere konstante Temperatur ein. Diese höhrere
Temperatur ist die Basis für die Messungen.


\subsection{Simulation des Experiments}
Da wir jetzt festgestellt haben, dass alle Teile individuell funktionieren, ist es Zeit sie zusammenzuführen, um das Experiment zu 
simulieren, um die Frage zu beantworten: Welchen Einfluss hat die Laserintensität auf die Schwerpunktsbewegung des Gases? Und welche 
Rolle spielt die Temperatur des Umgebenden Gases?\\
Zunächst müssen wir feststellen, ob die Anwendung des eHEX Algorithmus die Schwerpunktsbewegung des Nanoteilchens beeinflusst. Diese Untersuchung
resultiert in einer langweiligen Grafik mit einer konstanten, waagrechten Gerade, die zeigt, dass der eHEX Algorithmus alleine keinen 
Einfluss auf die Schwerpunktsbewegung hat.\\
Der nächste Schritt ist eine Art Kalibrierung. Bei Laserintensität $\Delta Q = 0$ und unter Einwirkung des Thermostaten sollte sich das
gesamte System auf die selbe Temperatur, nämlich die Temperatur des umgebenden Gases einstellen. Dies ist der Fall, wie in 
folgenden Grafiken zu sehen ist. Die gewählten Parameter sind: $P=0.8$, $T_\text{imp} = 0.05$, $T = 0.2$, $N=32$.\\
Da nun der Algorithmus kalibriert ist, werden die interessanteren Fälle von $\Delta Q > 0$ untersucht. Der eingestellte Druck 
ist proportional zur Anzahl der erzeugten Teilchen auf jeder Fläche. Höherer Druck bedeutet also mehr Teilchen, die Wärme vom 
Nanoteilchen abführen können, was zu einem höheren Maximalwert von $\Delta Q$ führt. Die Temperatur des Umgebenden Gases fließt ebenfalls 
in die Anzahl der erzeugten Teilchen ein, deshalb wurden zwei Temperaturen zur untersuchung gewählt: 0.05 und 0.1 - ein Viertel und die 
Hälfte der ursprünglichen internen Temperatur des Gasteilchens. Der Druck wird von 0.1 bis 1.2 in Schritten von 0.1 variiert und die 
zugeführte Wärmemenge von 0 bis 0.5 in Schritten von 0.05 im Allgemeinen (nicht konstant, sondern hängt von den Umständen ab, vor 
allem bei $P=0.1$ und $P=0.2$). Für die Darstellung werden 3 Werte für den Druck gewählt: $P=0.1$, $P=0.5$ und $P=1.0$. Für jeden 
dieser Druckwerte werden die Graphen für die Schwerpunktstemperatur, die interne Temperatur und die Temperatur des ausgehenden Gases 
geplottet. Dabei ist auf die Unterschiedliche Skalierung der Achsen zu achten.\\
Die Erste Beobachtung ist, dass die Schwerpunktsbewegung und die Temperatur des ausgehenden Gases mit steigendem $\Delta Q$ ansteigen.
Das ist zu erwarten, da bei höherer ins System gempumpter Wärme mehr Energie an die umgebenden Teilchen abgegeben wird. Weiters wird 
beobachtet, dass die Temperatur schneller für niedrige Drücke ansteigt (was durch die Unterschiedliche Skalierung der 
Ordinate nicht sofort auffällt). Dieses Verhalten ist ebenfalls zu erwarten, weil das Umgebende Gas als Thermostat wirkt und 
höherer Druck eine höhere Anzahl an Gasteilchen zur Folge hat, die mehr Wärme vom System abtransportieren.\\
Die Entwicklung der Temperatur hängt auch von der Temperatur des Umgebenden Gases ab. Die zur Temperatur $T=0.1$ korrespondieren (blau) 
steigen etwas schneller an, als die roten Kurven.\\
Millen et al. geben eine Formel für die Entwicklung der Schwerpunktstemperatur an. Diese Formel kann mit den Ergebnissen verglichen werden. 
Man sieht in der Grafik (die als Repräsentation gewählt wurde), 
dass die gemessene Temperatur schneller ansteigt als die berechnete. Dies kann einige Gründe haben. Ein Grund dafür könnte
ein Unterschiedlicher thermischer Akkomodationskoeffizient sein. Dieser ist gegeben durch
\begin{equation}
    \alpha = \frac{T_\text{em} - T_\text{imp}}{T_\text{sur} - T_\text{imp}}
\end{equation}
Der Akkomodationskoeffizient ist ein Maß für die übertragung der Energie vom Nanoteilchen auf das Gas. Der Wert für $\alpha$ im Paper von 
Millen et al. ist $\alpha = 0.777$, während der berechnete Wert für diese Simulation bei $\alpha = 0.016$. Das bedeutet, dass 
die Übertragung der Energie von Nanoteilchen auf Gasteilchen nicht auf die gleiche Art und Weise stattfindet. 



\subsection{Conclusio}
Um zusammenzufassen: Für diese Arbeit habe ich ein Optical Tweezer Experiment simuliert, um den Einfluss der Laserintensität 
auf die Schwerpunktsbewegung zu untersuchen. Die Ergebnisse zeigen, dass die Laserintensität, die Temperatur
des eingehenden Gases und der umgebende Druck die Schwerpunktsbewegung beeinflussen. 







\end{document}
